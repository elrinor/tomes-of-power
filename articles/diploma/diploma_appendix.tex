\newpage

% this is really retarded...

\renewcommand{\appendixtocname}{Приложение А. Список используемых терминов и обозначений}
%\renewcommand{\appendixpagename}{}

\appendix
%\appendixpage
\addappheadtotoc
\section*{Приложение А. Список используемых терминов и обозначений}
К сожалению, термины в области программирования не устоялись. В разных переводах, и в различных работах на русском языке, наблюдаются существенные рассогласования. Поэтому здесь приведены значения используемых в данной работе терминов. Все термины, имеющие отношения к языкам программирования, такие как {\it класс}, {\it объект}, или {\it функция}, рассматриваются в данной работе в применении к языку программирования Си++.

\begin{itemize}
\item {\it Абстрактный класс} (англ. {\it abstract class}) --- класс, содержащий хотя бы одну {\it чисто виртуальную функцию}.
\item {\it Адаптер} (англ. {\it thunk}) --- дополнительный код, сгенерированный компилятором, выполняющий преобразование (например, типов), и передающий управление какой-либо функции с использованием инструкции безусловного перехода.
\item {\it Базовый класс} (англ. {\it base class}). Класс \lstinline{B} является базовым классом для класса \lstinline{D}, если он является {\it непосредственным базовым классом} для класса \lstinline{D}, или {\it непосредственным базовым классом} для одного из базовых классов класса \lstinline{D} \cite{cpp03}. %[10]
\item {\it Бинарный интерфейс приложений} (англ. {\it application binary interface} ) --- набор конфигураций среды разработки и компилятора, гарантирующих бинарную совместимость разрабатываемых приложений. Бинарный интерфейс приложений определяет взаимодействие на низком уровне между приложениями, между компонентами приложения, между приложением и библиотеками, и между приложением и операционной системой на используемой платформе.
\item {\it Виртуальная функция} (англ. {\it virtual function}) --- метод класса, который может быть переопределён в производных классах так, что конкретная реализация метода для вызова будет определяться во время выполнения программы. В языке программирования Си++ виртуальные функции определяются с ключевым словом \lstinline{virtual} \cite{stroustrup97ru}.
\item {\it Виртуальный метод} (англ. {\it virtual method}) --- см. {\it виртуальная функция}.
\item {\it Встраивание} (англ. {\it inline expansion}) --- оптимизация, при которой вызов функции заменяется на подстановку копии ее тела.
\item {\it Декорирование имен} (англ. {\it name mangling}) --- декорирование имен добавляет дополнительную информацию к имени функций, структур, классов, и других типов данных, чтобы избежать конфликтов на этапе {\it компоновки} программы. Разные компиляторы используют разные схемы декорирования имен.
\item {\it Динамический полиморфизм} (англ. {\it dynamic polymorphism}) --- форма {\it полиморфизма}, подразумевающая выбор необходимой операции обработки данных на этапе выполнения программы, то есть с использованием {\it позднего связывания}. В языке Си++ динамический полиморфизм реализуется системой {\it виртуальных функций}.
\item {\it Динамическое связывание} (англ. {\it dynamic binding}) --- определение реализации вызываемого метода во время выполнения программы на основании фактического типа объекта (или объектов), для которого был вызван этот метод.
\item {\it Информация о типах времени выполнения} (англ. {\it run-time type information})--- стандартизованный интерфейс, с помощью которого программы на языке Си++, использующим указатели или ссылки на базовые классы, могут выяснять фактические типы объектов производных классов, к которым относятся эти указатели или ссылки \cite{lippman07ru}. Реализации информации о типах времени выполнения в разных компиляторах существенно отличаются.
\item {\it Компоновка} (англ. {\it linking}) --- процесс создания исполняемого модуля программы из одного или нескольких объектных модулей, полученных в результате компиляции исходных файлов программы.
\item {\it Наследник} (англ. {\it child class}) --- см. {\it производный класс}.
\item {\it Непосредственный базовый класс} (англ. {\it direct base class}). Класс \lstinline{B} является непосредственным базовым классом для класса \lstinline{D}, если в соответствующей программе на языке Си++ класс \lstinline{B} присутствует в списке базовых классов для класса \lstinline{D} \cite{cpp03}.%[10]
\item {\it Обратное проектирование} (англ. {\it reverse engeneering}) --- процесс извлечения информации о строении и принципах работы системы путем анализа ее функций, структуры, и поведения. В применении к программному обеспечению целью обратного проектирования является повышение уровня абстракции представления программ.
\item {\it Перегрузка функций} (англ. {\it function overloading}) --- возможность языка программирования поддерживать одновременное существование в одной области видимости нескольких различных вариантов функции, имеющих одно и то же имя, но различающихся типами параметров, к которым она применяется.
\item {\it Перекрытие методов} (англ. {\it method overriding}) --- особенность языка программирования, позволяющая производному классу заменить реализацию {\it виртуального метода} базового класса.
\item {\it Позднее связывание} (англ. {\it late binding}) --- см. {\it динамическое связывание}.
\item {\it Полиморфизм} (англ. {\it polymorphism}) --- в применении к языкам программирования, это возможность работать с данными разных типов с использованием одного и того же интерфейса.
\item {\it Полиморфный класс} (англ. {\it polymorphic class}) --- класс, имеющий хотя бы одну {\it виртуальную функцию} \cite{stroustrup97ru}.
\item {\it Предок} (англ. parent class) --- см. {\it базовый класс}.
\item {\it Производный класс} (англ. {\it derived class}). Класс \lstinline{D} является производным от класса \lstinline{B} классом, тогда и только тогда, когда класс \lstinline{B} является {\it базовым классом} для класса \lstinline{D}.
\item {\it Соглашение о вызовах} (англ. {\it calling conventions}) --- часть {\it бинарного интерфейса приложений}, которая регламентирует технические особенности вызова подпрограмм, передачи параметров, возврата из подпрограмм и передачи результата вычислений в точку вызова.
\item {\it Статический полиморфизм} (англ. {\it static polymorphism}) --- форма {\it полиморфизма}, подразумевающая выбор необходимой операции обработки данных на этапе компиляции программы, то есть с использованием {\it статического связывания}. Примерами статического полиморфизма в языке Си++ являются {\it шаблоны} и {\it перегруженные функции}.
\item {\it Статическое связывание} (англ. {\it static binding}) --- в применении к вызовам функций в программе на языке Си++, применение статического связывания подразумевает, что реализация вызываемой функции известна во время компиляции программы. Это в частности означает, что вызываемая функция может быть {\it встроена} в код вызывающей функции.
\item {\it Стековый фрейм} (англ. {\it stack frame}) --- зависимая от используемой платформы и компилятора структура данных, хранящая информацию о состоянии выполняемой подпрограммы.
\item {\it Таблица виртуальных функций} (англ. {\it virtual function table}) --- система, используемая в языках программирования для реализации {\it динамического связывания}. При использовании таблицы виртуальных функций, каждый объект {\it полиморфного класса} содержит указатель на соответствующую классу таблицу виртуальных функций, включающую адреса всех {\it виртуальных методов} класса \cite{stroustrup90ru}.
%\item {\it Объединение} (англ. union) --- тип данных \lstinline{union} языка Си++.
\item {\it Чисто виртуальная функция} (англ. {\it pure virtual function}) --- {\it виртуальная функция} без реализации. Реализации ее, возможно, различные, могут содержаться в производных классах.
%\item {\it Шаблоны} (англ. {\it templates}) — средство языка C++, предназначенное для реализации обобщенных алгоритмов, без привязки к некоторым параметрам, таким как типы данных, размеры буферов, значения по умолчанию, и т.п.
%\item {\it Curiously Recurring Template Pattern} --- прием, при использовании которого объявляемый класс \lstinline{Derived} наследуется от шаблонного класса, инстанциированного с \lstinline{Derived} в качестве одного из параметров шаблона. Чаще всего используется для реализации формы статического полиморфизма, схожей с используемой в системе виртуальных функций языка Си++, но использующей статическое связывание.
\end{itemize}


Ниже перечислены основные обозначения, используемые в данной работе.
\begin{itemize}
\item $R^{-1}$ --- отношение, обратное к $R \subseteq X \times X$. $R^{-1} = \{(y, x) | (x, y) \in R \}$.
\item $R^+$ --- транзитивное замыкание бинарного отношения $R \subseteq X \times X$. $R^+$ является наименьшим транзитивным отношением на множестве $X$, включающим $R$.
\item $R^=$ --- рефлексивное замыкание бинарного отношения $R \subseteq X \times X$. $R^=~=~R \cup \{ (x, x) | x \in X \}$.
\item $R^{\ne}$ --- рефлексивное сужение бинарного отношения $R \subseteq X \times X$. $R^{\ne}~=~R~\setminus~\{ (x, x) | x \in X \}$.
%\item $R^s$ --- симметричное замыкание бинарного отношения $R \subseteq X \times X$. $R^s~=~{(x, y) | (x, y) \in R \vee (y, x) \in R}$.
\item $S \circ R$ --- композиция бинарных отношений $R, S \subseteq X \times X$, определенная как $S \circ R = \{(x, z) | \exists y \in X: (x, y) \in R \wedge (y, z) \in S\}$.
\item $|X|$ --- мощность множества $X$.
\item $\nG(X)$ --- множество всевозможных графов $G = (X, E)$.
\item $\nT(X)$ --- множество всевозможных деревьев $T = (X, E)$.
\item $\nT^R(X)$ --- множество всевозможных корневых деревьев $T = (X, E, r)$, где $r \in X$ --- корень дерева.
\item $R[Y]$ --- сужение отношения $R \subseteq X \times X$ на множество $Y \subseteq X$, то есть $R[Y] = R \cap (Y \times Y)$.
\item $T[Y]$ --- сужение дерева $T = (X, E)$ на множество вершин $Y \subseteq X$, то есть $T[Y] = (Y, E[Y])$.
\item $T[\nT]$ --- дерево $T' = (X, E)$, соответствующее корневому дереву $T = (X, E, r)$.
\item $\gC_{\bP}$ --- множество всех полиморфных классов программы на языке Си++ $\bP$.
\item $\leftarrow_{\bP}$ --- бинарное отношение непосредственного наследования, определенное на множестве $\gC_{\bP}$ для программы на языке Си++ $\bP$. Для любых двух классов $B, D \in \gC_{\bP}$, $B~\leftarrow_{\bP}~D \enspace \Longleftrightarrow \enspace B \enspace$ является {\it непосредственным базовым классом} для $\enspace D$.
\item $\lhd_{\bP} = \enspace \leftarrow_{\bP}^+$ --- бинарное отношение наследования, определенное на множестве $\gC_{\bP}$ для программы на языке Си++ $\bP$. Для любых двух классов $B, D \in \gC_{\bP}$, $B~\lhd_{\bP}~D \enspace \Longleftrightarrow \enspace B \enspace \text{является {\it базовым классом} для} \enspace D$. Это обозначение соответствует определению базового класса, данному выше.
\end{itemize}
