\newpage
\section*{Заключение}
\addcontentsline{toc}{section}{Заключение}
В данной работе были представлены методы автоматического восстановления объектных структур данных из низкоуровневого представления программ на языке Си++. Так как программа на языке Си++ может быть скомпилирована как с поддержкой, так и без поддержки информации о типах времени выполнения, то были рассмотрены оба случая.

Формат информации о типах времени выполнения зависит от используемого компилятора. Были рассмотрены форматы, используемые компиляторами GCC и MSVC. Для этих компиляторов также были рассмотрены основные положения используемых ими бинарных интерфейсов приложений, связанные с реализацией виртуальных функций. Анализ бинарных интерфейсов приложений показал, что задача поиска информации о типах времени выполнения сводится к задаче поиска таблиц виртуальных функций. Был представлен алгоритм поиска таблиц виртуальных функций, и метод восстановления иерархий классов, использующий этот алгоритм для локализации информации о типах времени выполнения.

Для случая отсутствия информации о типах времени выполнения был предложен метод поэтапного восстановления отношения наследования. Были рассмотрены правила языка программирования Си++ и положения бинарных интерфейсов приложений, используемых компиляторами GCC и MSVC, связанные с множественным наследованием. Анализ этих правил показал, что при отсутствии виртуальных деструкторов, множественное наследование может быть корректно заменено одиночным.

Было показано, что отношение наследования на множестве классов можно расширить на множество таблиц виртуальных функций, и что при этом наследование на множестве таблиц виртуальных функций является одиночным. Для восстановления одиночного наследования был разработан метод, основывающийся на анализе:
\begin{itemize}
\item Таблиц виртуальных функций;
\item Параметров виртуальных функций;
\item Вызовов виртуальных функций;
\item Виртуальных деструкторов.
\end{itemize}

Для восстановления множественного наследования был предложен метод объединения таблиц виртуальных функций, принадлежащих одному классу, и модификации деревьев одиночного наследования на множестве таблиц виртуальных функций, основывающийся на анализе виртуальных деструкторов.

Было разработано приложение crec, в котором были реализованы предложенные методы восстановления объектных структур данных как для случая присутствия информации о типах времени выполнения, так и для случая ее отсутствия. Приложение crec было протестировано на системе документирования исходных кодов doxygen, и были получены положительные результаты. В случае присутствия информации о типах времени выполнения иерархия полиморфных классов и таблицы виртуальных функций были восстановлены точно и полностью. В случае отсутствия информации о типах времени выполнения иерархия полиморфных классов была восстановлено корректно, с точностью до замены множественного наследования одиночным и удаления некоторых классов, для которых компилятором не были сгенерированы таблицы виртуальных функций.








